% !TeX spellcheck = de_DE
\documentclass[acmlarge,authorversion,nonacm]{acmart}
\usepackage[acronym]{glossaries}
\usepackage{todonotes}
\AtBeginDocument{%
  \providecommand\BibTeX{{%
    Bib\TeX}}}


%\setcopyright{acmcopyright}
\copyrightyear{2024}
\acmYear{2024}
\acmDOI{XXXXXXX.XXXXXXX}


%%
%% These commands are for a JOURNAL article.
\acmJournal{JACM}
\acmVolume{37}
\acmNumber{4}
\acmArticle{28}
\acmMonth{2}

\newacronym{toc}{TOC}{Theory of Contraints}
\newacronym{llm}{LLM}{Large Language Model}

\begin{document}
	\title{ChangeMangement - Vergleich von Kaizen und Theory of Constraints}
	
	\author{Sebastian Heiden}
	\email{u38439@hs-harz.de}
	\affiliation{%
		\institution{Hochschule Harz}
		\streetaddress{Friedrichstrasse 57-59}
		\city{Wernigerode}
		\state{Sachsen-Anhalt}
		\country{Deutschland}
		\postcode{38855}
	}

%%
%% The abstract is a short summary of the work to be presented in the
%% article.
\begin{abstract}

\end{abstract}


\received{31 Januar 2024}


%%
%% This command processes the author and affiliation and title
%% information and builds the first part of the formatted document.
\maketitle

\section{Einleitung und Zielstellung}

Wir wollen in dieser Arbeit die beiden die beiden Change Mangement-Ansätze Kaizen und \gls{toc} gegenüberstellen.
Gleichzeitig soll die Leistungsfähigkeit von \glspl{llm} zum schreiben von naturwissenschaftlichen Fachaufsätzen in diesem Szenario getestet und ausgereizt werden.
\todo{weitere Erläuterungen von Kaizen und TOC mit Quellen.}
\section{Methoden}\label{sec:methods}
% !TeX spellcheck = de_DE

Ausgehend von den Voruntersuchungen mit ChatGPT 3.5 und ChatGPT 4, soll halb automatisiert ein Fachartikel erstellt werden,
der die Methoden Kaizen und \gls{toc} anhand eines Beispiels entlang der Prozesskette/Nutzung vergleicht~\cite{openai_chatgpt_nodate}. 
Der Text wird mit deutschen Prompts auf Deutsch erzeugt. Auf Basis von Voruntersuchungen mit ChatGPT 3.5 und 4, soll der Aufsatz dann mit ChatGPT entworfen werden. Inklusive der Forderung einer \LaTeX{}-Formatierung und Referenzen mit \BibTeX{} anzugeben.

Ziel ist es, dass die \gls{ki} möglichst viel an einem Stück entwirft, sodass Fragmente nicht zusammen getragen werden muss. Mikro-Management soll durch das Prompten von Texten für einzelne Absätzen vermieden werden. Da dies bei der Generierung aufwendiger ist auch mehr Fachwissen bedarf, die erst durch Vorarbeiten angeeignet werden sollen. Ggf. fügen wir dennoch Absätze des \gls{ki}-Entwurfs auf der Teilsatzebene zusammen. Wenn passend, werden auch die Teilsätze aus anderen Kapiteln übernommen. Bindewörter des Autoren sind durch eckige Klammern gekennzeichnet.

Danach wird dieser Text vollständig von einer Sprach in eine Zielsprache mit Deepl Pro übersetzt~\cite{deepl_deepl_nodate}.
Die einzelnen Übersetzungsschritte sind:

\begin{enumerate}
	\item Deutsch $\rightarrow$ Französisch
	\item Französisch $\rightarrow$ Russisch
	\item Russisch $\rightarrow$ Englisch
\end{enumerate}

Es handelt sich hierbei um eine qualitative Untersuchung. Es soll die Qualität des Entwurfs des deutschen Texts und die Fähigkeit der Übersetzungen ermittelt werden. Dies wurde gezielt so gewählt, um die Leistungsfähigkeit der Systeme zu fordern. Da weniger deutsche als englische Text im Internet verfügbar sind~\cite{noauthor_sprachen_nodate}, soll gezielt ein deutscher Text generiert werden unter der Prämisse, dass deutsche Text nach unserer Erfahrung inhaltlich schlechter ausfallen. Ein Textvergleich mit jeweils englischen und deutschen Prompts, wird aufgrund des Aufwands vermieden. Es sollen insgesamt gezielt Unzulänglichkeiten der \gls{llm} untersucht werden.

Es werden gezielt mehre Übersetzungsstufen verwendet, wie die Güte der Übersetzung je nach Sprachkombination variiert. Alle Prompts und Antworten, sowie alle Übersetzungen sind im Appendix angegeben. Werden Prompts korrigiert, sei es durch Fehler bei der Frage, oder weil die Fragestellung nicht zum wünschten Ergebnis führen, wird dieser Pfad nicht im Export des Chat und somit auch nicht hier im Anhang aufgeführt.
Der deutscher Text soll eine Ziellänge von vier Seiten bei gleicher Formatierung erreichen. Der englische Text kann dadurch etwas kürzer ausfallen. Der deutsche und der englische Text wird dieser Arbeit vorangestellt. 
Der englische Text wird auf formelle als auch inhaltliche Korrektheit auf dem Level eines wissenschaftlichen Fachaufsatzes überprüft. Der deutsche, der korrigierte französische und korrigierte russische Text wird dem Anhang beigefügt.


\section{Ergebnisse}\label{sec:results}
% !TeX spellcheck = de_DE

\subsection{Beschreibung der Voruntersuchungen}

In den Voruntersuchungen wurden einmal ein ChatGPT 3.5 (im October 2023) und ChatGPT 4.0 (Im Januar 2024)
einen Aufsatz zum Vergleich von Kaizen und \gls{toc} zu entwerfen.

Bei der Untersuchung von ChatGPT 3.5(siehe Anhang~\ref{sec:voruntersuch3}) wurde der Frageinhalt voll verstanden. ChatGPT weigerte sich einen längeren Text zu entwerfen.
Bei der Nachfrage welche Unternehmen Kaizen und \gls{toc} anwenden, wurden für Kaizen Toyota, General Electric, Ford, Nestlé  für \gls{toc}: Amazon, Intel, General Electric, Ford, Boing angegeben. Bei der expliziten Fragen, welche Unternhemen beides anwenden werden Toyota, General Electric, Procter \& Gamble aufgelistet.

Als ChatGPT 4, gefragt wurde einen Aufsatz zu entwerfen, wurde nach einen Grobentwurf nachgefragt, ob der Artikel in der Art geschrieben werden solle (siehe Anhang~\ref{sec:voruntersuch4}). Bei der Nachfrage nach Firmen, die beide Techniken verwenden, wurde eine Netzsuche mit \textit{BING} gestartet. Dabei kam es zum Missverständnis, dass \gls{toc} als total organic carbon missverstanden wurde. Was nur aus dem Weblinks ersichtlich ist, die Referenziert wurden. Daher konnten, keine Firmen gefunden werden. Danach haben wir ChatGPT gebeten Unternehmen in der Automobilbranche zu finden, die eine der beiden Methoden benutzten. Dabei wurde nur Prosche gefunden. Danach wurde die \gls{ki} gebeten mit einen hypotetischen Unternehmen in der Automobilbranche das beide Techniken verwendet und jeweils einem Unternehmen, dass beide Verfahren verwendet, einen Aufsatz zu entwerfen. In der Antwort gibt ChatGPT 4, dass es kein Unternehmen in der Automobilbranche findet, dass \gls{toc} anwendet. Im folgenden wird ein sehr kurzer Ansatz mit nur noch dem Hypothetischen Fall angeboten und fragt, nach damit der Finale Text entworfen werden soll. Der Text ist nur eine der vier Seiten lang. Quellen werden nicht in einem Literaturverzeichnis, sondern als hochgestellten Weblink angegeben.

\subsection{Beschreibung des KI-Textes: Erstellung und Inhalt}

Mit diesem Voruntersuchungen, soll nun mit ChatGPT 4 versucht werden, einen Aufsatz zum Vergleich von Kaizen und \gls{toc} zu erstellen. 



\section{Diskussion}\label{sec:discussion}
% !TeX spellcheck = de_DE




\section{Schlussfolgerung}\label{sec:conclusion}


\begin{acks}
Vielen Dank an Katerina Shapiro-Heiden für die Untersuchung der automatisierten Übersetzungen.
\end{acks}

\bibliographystyle{ACM-Reference-Format}
\bibliography{sample-base}

\newpage
%%
%% If your work has an appendix, this is the place to put it.
\appendix

\section{Appendix}

\subsection{Prompts und Antworten der Voruntersuchung}
% !TeX spellcheck = de_DE
\setlength\parskip{0.5em plus 0.1em minus 0.2em}
\setlength\parindent{0pt}

Folgender Dialog wurde zwischen dem Benutzer und ChatGPT 3.5 Ende Oktober 2023 durchgeführt. 

\textbf{TOC vs. Kaizen in Change}


\textbf{user}

Entwerfen einen Aufsatz der die Unterschiede und Gemeinsamkeiten von TOC und Kaizen im Sinne des Change Management darstellt. Kann man beide zu einer Methode zusammenfügen? Der Text muss vier Seiten lang sein. 

\textbf{ChatGPT}

Verstehe, dass Sie nach einem Aufsatz suchen, der die Unterschiede und Gemeinsamkeiten von Theory of Constraints (TOC) und Kaizen im Kontext des Change Managements darstellt. Die Erstellung eines vierseitigen Aufsatzes erfordert jedoch eine umfassende Analyse und Beschreibung, die hier nicht vollständig durchgeführt werden kann. Ich kann jedoch eine kurze Übersicht über die Unterschiede und Gemeinsamkeiten bieten, um Ihnen als Ausgangspunkt zu dienen:

\textit{Einführung}:
- Einführung in TOC und Kaizen als Konzepte im Change Management.

\textit{Unterschiede:}
- TOC ist eine Methodik zur Identifizierung und Beseitigung von Engpässen, während Kaizen sich auf kontinuierliche Verbesserung in allen Prozessen konzentriert.
- TOC betont die Bedeutung der Identifizierung des kritischsten Engpasses, während Kaizen auf kleine, schrittweise Verbesserungen in allen Bereichen abzielt.
- TOC legt den Fokus auf die Maximierung der Durchsatzrate, während Kaizen auf die Verbesserung von Effizienz und Qualität abzielt.

\textit{Gemeinsamkeiten:}
- Beide Ansätze fördern die Zusammenarbeit und Kommunikation innerhalb des Unternehmens.
- Sowohl TOC als auch Kaizen erfordern ein starkes Engagement für den kontinuierlichen Verbesserungsprozess.
- Sie können beide zur Lösung von Problemen und zur Steigerung der Wettbewerbsfähigkeit eingesetzt werden.

\textit{Kann man beide zu einer Methode zusammenfügen?}
Ja, in der Praxis können TOC und Kaizen kombiniert werden, um ein effektives Change-Management-Tool zu schaffen. Zum Beispiel kann TOC verwendet werden, um die kritischsten Engpässe zu identifizieren, und Kaizen kann anschließend angewendet werden, um schrittweise Verbesserungen in diesen Bereichen vorzunehmen. Dies ermöglicht eine umfassende Optimierung von Prozessen.

In einem vierseitigen Aufsatz könnten Sie diese Punkte ausführlicher erläutern, Beispiele aus der Praxis geben und auf potenzielle Herausforderungen bei der Integration von TOC und Kaizen eingehen. Bitte lassen Sie mich wissen, ob Sie weitere Informationen oder Hilfe bei der Ausarbeitung des Aufsatzes benötigen.

\textbf{user}

Welche Beispiele gibt es den? 

\textbf{ChatGPT}

Natürlich, hier sind einige Beispiele, wie Theory of Constraints (TOC) und Kaizen im Change Management kombiniert werden können:

1. \textit{Produktionsprozesse:} Nehmen wir an, ein Produktionsunternehmen verwendet TOC, um Engpässe in seiner Fertigungslinie zu identifizieren. Der kritischste Engpass ist die Montageabteilung. Hier kann Kaizen angewendet werden, um die Montageprozesse kontinuierlich zu verbessern. Durch die Implementierung von Kaizen-Prinzipien, wie der Reduzierung von Verschwendung und ineffizienten Schritten, kann die Montagezeit verkürzt und die Produktionseffizienz gesteigert werden.

2.\textit{ Lieferkettenmanagement:} In einer komplexen Lieferkette kann TOC verwendet werden, um den Engpass zu ermitteln, der die Lieferungen verzögert. Angenommen, es handelt sich um einen Zulieferer, der Schwierigkeiten bei der rechtzeitigen Bereitstellung von Komponenten hat. Hier kann Kaizen eingesetzt werden, um die Kommunikation und die Zusammenarbeit mit dem Zulieferer zu verbessern. Schrittweise Verbesserungen können zu einer besseren Koordination führen und die Lieferzuverlässigkeit erhöhen.

3. \textit{Dienstleistungssektor:} In einem Serviceunternehmen kann TOC helfen, den Engpass in der Kundenbetreuung zu identifizieren, beispielsweise durch lange Wartezeiten bei Kundenanfragen. Mit Kaizen können die Mitarbeiter kontinuierlich daran arbeiten, die Kundenbetreuungsprozesse zu optimieren. Das könnte die Implementierung von Self-Service-Optionen für Kunden oder die Schulung der Mitarbeiter in effizienteren Arbeitsweisen einschließen.

4. \textit{Gesundheitswesen:} Im Gesundheitswesen kann TOC dazu beitragen, Engpässe in der Patientenversorgung zu erkennen, wie lange Wartezeiten in der Notaufnahme. Kaizen kann hier genutzt werden, um Prozesse zu optimieren, die zu schnelleren Diagnosen und Behandlungen führen. Dies könnte die Einführung eines effizienteren Patienten-Trackings oder die Schulung des medizinischen Personals in effektiverer Patientenbetreuung beinhalten.

Diese Beispiele verdeutlichen, wie TOC und Kaizen in verschiedenen Geschäftsbereichen zusammengeführt werden können, um Engpässe zu identifizieren und kontinuierliche Verbesserungen in den relevanten Prozessen vorzunehmen. Die Kombination dieser Ansätze kann dazu beitragen, Effizienz und Qualität zu steigern, Kosten zu senken und die Kundenzufriedenheit zu verbessern.

\textbf{user}

Welche Firmen verwenden diese Ansätze? 
 
\textbf{ChatGPT}

Viele Unternehmen auf der ganzen Welt haben Theory of Constraints (TOC) und Kaizen erfolgreich in ihren Betriebs- und Change-Management-Praktiken implementiert. Hier sind einige Beispiele von Unternehmen, die diese Ansätze verwenden:

\textit{1. Toyota:} Toyota ist bekannt für die Implementierung von Kaizen in seinen Produktionsprozessen. Das Unternehmen war Pionier in der Anwendung von kontinuierlicher Verbesserung und effizienten Produktionsmethoden, die als "Toyota Production System" bekannt sind.

\textit{2. Amazon:} Amazon nutzt TOC-Prinzipien in seinen Lager- und Logistikzentren, um Engpässe zu identifizieren und zu beheben. Dies ermöglicht es Amazon, seine Lieferzeiten zu optimieren und die Kundenzufriedenheit zu steigern.

\textit{3. Intel:} Intel hat TOC-Prinzipien in seiner Fertigung eingesetzt, um Engpässe in der Produktion von Mikrochips zu erkennen und die Produktionseffizienz zu steigern.

\textit{*4. General Electric (GE):} GE hat TOC und Kaizen verwendet, um seine Prozesse in verschiedenen Geschäftsbereichen, einschließlich der Fertigung von Flugzeugtriebwerken, zu optimieren.

\textit{5. Ford:} Ford hat sowohl TOC als auch Kaizen in seinen Fertigungsprozessen eingesetzt, um die Effizienz und Qualität in der Automobilproduktion zu steigern.

\textit{6. Nestlé:} Nestlé hat Kaizen in seinen Produktionsprozessen implementiert, um die Qualität und Effizienz in der Lebensmittelherstellung zu verbessern.

\textit{7. Boeing:} Boeing hat TOC-Prinzipien in der Flugzeugproduktion verwendet, um die Produktionskosten zu senken und die Fertigungszeiten zu verkürzen.

Diese Unternehmen haben die Vorteile von TOC und Kaizen erkannt und erfolgreich in ihren Betriebsabläufen integriert, um kontinuierliche Verbesserungen zu erzielen, Engpässe zu identifizieren und Prozesse effizienter zu gestalten. Es ist wichtig zu beachten, dass viele Unternehmen, unabhängig von ihrer Branche, diese Ansätze anpassen und auf ihre spezifischen Bedürfnisse zuschneiden können.

\textbf{user}

Gibt es Firmen die beides Nutzen? 

\textbf{ChatGPT}

Ja, es gibt Unternehmen, die sowohl Theory of Constraints (TOC) als auch Kaizen in ihren Geschäftspraktiken und im Change Management kombinieren. Die Kombination beider Ansätze ermöglicht es Unternehmen, Engpässe zu identifizieren und zu beheben, während gleichzeitig kontinuierliche Verbesserungen in ihren Prozessen vorangetrieben werden. Hier sind einige Beispiele:

\textit{1. Boeing:} Boeing ist ein Beispiel für ein Unternehmen, das sowohl TOC als auch Kaizen verwendet. In der Flugzeugproduktion hat Boeing TOC-Prinzipien eingesetzt, um Engpässe zu identifizieren und die Produktionszeiten zu verkürzen. Gleichzeitig hat das Unternehmen Kaizen-Methoden in verschiedenen Bereichen der Produktion angewendet, um die Qualität und Effizienz zu steigern.

\textit{2. Toyota:} Toyota ist bekannt dafür, sowohl TOC- als auch Kaizen-Prinzipien in seinen Produktionsprozessen zu nutzen. Das Toyota Production System (TPS) ist eine Kombination aus TOC und Kaizen, die kontinuierliche Verbesserung und Engpassmanagement in der Automobilproduktion ermöglicht.

\textit{3. General Electric (GE):} GE hat sowohl TOC als auch Kaizen in verschiedenen Geschäftsbereichen eingesetzt. In der Fertigung von Flugzeugtriebwerken wurde TOC verwendet, um Engpässe zu beseitigen, während Kaizen zur kontinuierlichen Verbesserung beitrug.

\textit{4. Procter \& Gamble (P\&G):} P\&G ist ein Unternehmen, das sowohl TOC als auch Kaizen in seinen Produktions- und Supply-Chain-Prozessen verwendet, um Engpässe zu identifizieren und kontinuierliche Verbesserungen in der Produktion und Logistik voranzutreiben.

Diese Unternehmen haben erkannt, dass die Kombination von TOC und Kaizen eine effektive Strategie sein kann, um sowohl Engpässe zu bewältigen als auch eine Kultur kontinuierlicher Verbesserung zu fördern. Die spezifische Implementierung kann je nach den Anforderungen und Zielen des Unternehmens variieren.

\subsection{Prompts und Antworten der Untersuchung}



\end{document}
\endinput