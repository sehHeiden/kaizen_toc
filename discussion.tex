% !TeX spellcheck = de_DE

\todo{Voruntersuchung: Recherche verwenden die Genanten Unternehmen wirklich Kaizen und TOC?}
Bei der Voruntersuchung hat nur ChatGPT 3.5 eine Anzahl von unternehmen gefunden, die eine oder beide Methoden verwenden. Die Verschlechterung bei ChatGPT 4 kann durch die Verwendung der \textit{BING}-Suche entstehen. Da die Suche nur auf Deutsch erfolgte, war die Quellenlage womöglich nicht ausreichend.
Dafür spricht, dass Toyota nur indirekt als Unternehmen gelistet wird, dass Kaizen verwendet. Indem erwähnt wird, dass Toyota personal Porsche in Kaizen schulte.
Daher haben wir darauf wert gelegt, im folgenden Suchergebnisse in mehreren Sprachen zuzulassen.

Im allgemeinen sind die Antworten der \gls{ki} von durchwachsener Qualität. Wie haben oft die Wahrnehmung gehabt, dass Dinge, die die \gls{ki} angab nicht zu können, dann doch konnte. Das extremste Ausmaß ist hier die Behauptung keine Websuche machen zu können, um diese dann eine Seitenlänge später zu machen. Auch sehr deutlich ist der Hinweis ein Zugriff auf Fachartikel und Statistiken zu haben, dann Tage später darauf hinzuweisen, dass man nach Fallstudien suchen müsse, die dann erfolgreich referenziert werden könnte.
Allgemein scheint es etwas wie eine Tagesform zu geben. Die Frage, inwieweit der bisherige Verlauf als Kontext eingelesen wird und wie sich das vom Livechatverlauf als Kontext unterscheidet. Das im Verlauf des Tages der alte Verlauf als Trainingsmaterial herangezogen wird scheint unwahrscheinlich. Wenn der Chat am Tag darauf fortgesetzt wurde, schien es, dass die \gls{ki} sich anders verhält. An den ersten Tagen wurde nur mit Weblinks und eckigen Klammern indirekt zitiert. Am Ende der Untersuchung wurde ein direkten Zitierstil verwendet. Das kann dazu beitragen, dass der \gls{ki}-Text uneinheitlich erscheint.
Verwunderlich ist auch warum generell die Verwendung von Wissenschaftlichen Quellen auch bei der Suche erschwert ist, da es doch bisher eine Reihe von Prerelease-Servern und Open Access-Publikationen existieren. Sind bei klassischen Publikationen im die Zusammenfassungen häufig verfügbar. Es scheint uns unwahrscheinlich, dass all diese Resourcen sowohl für die Suchindizierung und das Training ausgeschlossen worden sind.

Der Textstil wirkt uneinheitlich. So schreibt die \gls{ki} bei den Einleitungsteilen sehr ausladende Texte. Bei Neuformulierung mit der Bitte um mehr Inhalt, wird genauso viel Text mit fast identischem Inhalt in fast gleicher Länge geschrieben. Verschärft wird der Eindruck für den Leser durch manuelle Zusammentragen der Textfragmente. 
Auch in der Zusammenfassung wurde durch die\gls{ki} ein Vergleichbarer Text geschrieben. Dadurch entsteht, ein Text der für eine Zusammenfassung ungewöhnlich lang ist und nicht auf die Schlussfolgerung einging. Dies lässt sich aber, durch neue Prompts, beheben. Bei der Darstellung von Webreferenzen für den wissenschaftlichen Teil fallen, die Texte aber im Vergleich kurz aus, obwohl und die Ableitungen daraus, den wissenschaftlichen Kern der Arbeit darstellen. Interessante Aspekte wurden bei~\cite{ibscmr2003toyotakaizen} weggelassen.
 
 \todo{Kontrolle der Quellen}

Es wurde eine Rein qualitative Darstellung durchgeführt, da nur ein einzelner Text und darüber hinaus keine Vergleichsbasis in Sinne eines menschlich geschrieben Textes erstellt wird. Zwar gibt es im Zusammenhang mit \gls{llm} gibt es einige Studien \todo{Quelle?} die Produktivitätsgewinne untersuchen. Es stellt sich im allgemeinen die Frage, mit welchem Niveau an Vorwissen und Erfahrung beim Schreibenden des Vergleichstextes anwenden sollte. Im Idealfall werden Texte von verschiedenen Niveaus wie Schüler, Studenten, Berufstätige, Forschende gepromptet und als Vergleich geschrieben. Dabei dürfte der Promptende\todo{Wie schreiben} nicht die selbe Person sein, die einen Vergleichstext schreibt, um Leaks in der Form von Wissen (Zeitersparnis) und Erwartbarkeit des Ergebnisses auszuschließen.\todo{In welches Kapitel kommt dieser Absatz?}
