% !TeX spellcheck = de_DE

Bei der Voruntersuchung hat nur ChatGPT 3.5 eine Anzahl von Unternehmen gefunden, die eine oder beide Methoden verwenden. Eine Websuche bestätigt diese Ergebnisse, zum Beispiel General Eletric und Procter \& Gamble~\cite{noauthor_heat_nodate,noauthor_what_nodate}. Die Verschlechterung bei ChatGPT 4 kann durch die Verwendung der \textit{BING}-Suche entstehen. Da die Suche nur auf Deutsch erfolgte, war die Quellenlage womöglich nicht ausreichend. Wir gehen daher davon aus, das eine Websuche sogar eigentliches Wissen überdecken kann.
Dafür spricht, dass Toyota nur indirekt als Unternehmen gelistet wird, das Kaizen verwendet, indem erwähnt wird, dass Toyota Porsche in Kaizen schulte~\cite{noauthor_kaizen_nodate}. Daher haben wir darauf Wert gelegt, im Folgenden Suchergebnisse in mehreren Sprachen zuzulassen.

Im Allgemeinen sind die Antworten der \gls{ki} von unterschiedlicher Qualität. Wir haben oft die Wahrnehmung gehabt, dass Dinge, die die \gls{ki} angab nicht zu können, dennoch möglich waren. Das extremste Ausmaß ist hier die Behauptung keine Websuche machen zu können, um diese dann eine Seitenlänge später selbst durchzuführen. Auch sehr deutlich ist der Hinweis keinen Zugriff auf Fachartikel und Statistiken zu haben, dann Tage später auf den Suchbegriff Fallstudien hinzuweisen, die dann erfolgreich referenziert werden konnten.
Allgemein scheint es etwas wie eine Tagesform zu geben. Dies eröffnet die Frage, inwieweit der bisherige Verlauf als Kontext eingelesen wird und wie sich das vom Livechatverlauf als Kontext unterscheidet. Dass im Verlauf des Tages der alte Verlauf als Trainingsmaterial herangezogen wird, scheint unwahrscheinlich. Wenn der Chat am Tag darauf fortgesetzt wurde, schien es, dass die \gls{ki} sich anders verhält und bessere Lösungsvorschläge liefert.
Verwunderlich ist auch, warum generell die Verwendung von wissenschaftlichen Quellen bei der Suche erschwert ist, obwohl eine Reihe von Prerelease-Servern und Open Access-Publikationen existiert, Außerdem sind bei klassischen Publikationen häufig Zusammenfassungen im Internet verfügbar. Es scheint uns unwahrscheinlich, dass all diese Ressourcen sowohl für die Suchindizierung, als auch für das Training ausgeschlossen worden sind.

An den ersten Tagen wurde nur mit Weblinks und eckigen Klammern indirekt zitiert. Am Ende der Untersuchung wurde ein direkter Zitierstil verwendet. Dies kann dazu beitragen, dass der \gls{ki}-Text uneinheitlich erscheint. Des Weiteren schreibt die \gls{ki} in der Einleitung sehr ausladende Texte, mit allgemeinem Inhalt. Bei Neuformulierung mit der Bitte um mehr/konkreteren Inhalt, wird genauso viel Text mit fast identischem Inhalt in fast gleicher Länge geschrieben. Verschärft wird der Eindruck für den Leser durch manuelles Zusammentragen der Textfragmente. 
Auch in der Zusammenfassung wurde durch die \gls{ki} ein vergleichbarer Text geschrieben. Dadurch entsteht ein Text, der für eine Zusammenfassung ungewöhnlich lang ist und nicht auf die Schlussfolgerung einging. Dies lässt sich aber durch neue Prompts beheben. Bei der Darstellung von Webreferenzen für den wissenschaftlichen Teil fallen die Texte aber im Vergleich kurz aus, obwohl die Ableitungen daraus den wissenschaftlichen Kern der Arbeit darstellen. Interessante Aspekte wurden beim Zitieren von \textit{"`Toyota's Kaizen Experience"'}~\cite{ibscmr2003toyotakaizen} weggelassen.
 
Alle genannten Quellen existieren und sind nicht halluziniert, was durch die Websuche der \gls{ki} sichergestellt wurde. Auch wir können dies bestätigen. Dennoch sind zwei der neun Weblinks in der \BibTeX{}-Datei falsch. Die KI hat hier falsche Links angegeben, obwohl die echten bekannt waren. Zudem wurden in vier der neun Quellen Institute und Magazine als Autoren angeben. Der Inhalt von vier Quellen konnte nicht vollumfänglich geprüft werden, da es keinen Zugriff auf diese Quellen gab. Zumindest zwei der vier Zitationen scheinen nach verfügbarer Zusammenfassung korrekt. Die zwei verbleibenden, ungeklärten Quellen sind Fachbücher~\cite{imai1986kaizen, womack1996lean}. Mindestens zwei der verwendeten Zitate sind falsch. Zwar existieren die Quellen, sie beschäftigen sich auch mit der genannten Thematik, aber nicht mit dem detaillierten Inhalt, der im jeweiligen Absatz beschrieben wurde. 
Es gelang uns nicht, Fallstudien zur Verwendung von \gls{toc} bei Toyota zu finden. Allerdings können wir eine Vielzahl von wissenschaftliche Publikation zu \gls{toc} wie Modellrechnungen~\cite{gundogar_dynamic_2016, thurer_bottleneck-oriented_2018} und Fallstudien~\cite{umble_implementing_2006} finden. 

Es wurde eine rein qualitative Untersuchung durchgeführt, da nur ein einzelner Text und darüber hinaus keine Vergleichsbasis in Sinne eines menschlich geschriebenen Textes erstellt wurde. Es gibt im Zusammenhang mit \gls{llm} Studien, die Produktivitätsgewinne untersuchen und finden~\cite{noy_experimental_2023}. Es stellt sich im Allgemeinen die Frage, über welches Niveau an Erfahrung und Vorwissen der/die Schreibende des menschlichen Vergleichtextes verfügen sollte. Im Idealfall werden Texte von verschiedenen Kompetenzniveaus wie Schülerïnnen, Studentïnnen, Berufstätige, Forschende gepromptet und als Vergleich geschrieben. Dabei sollte der Promptende nicht auch Autor des Vergleichstexts sein, um Leaks in der Form von Wissen (Zeitersparnis) und Erwartbarkeit des Ergebnisses auszuschließen.


Bei der Übersetzung treten sehr wenige Fehler auf. Problemfelder sind anscheinend Namen und kurze Abkürzungen. Namen, die an eine Sprache angeglichen werden, werden bei der zweiten Übersetzung des Namens nicht, wieder in die ursprüngliche Form gebracht. Die Abkürzung sollte in allen Sprachen gleich sein. Der auffälligste Fehler ist das Fehlen eines ganzes Satzes. Dabei handelt es sich auch nicht um eine Zusammenfassung. Dies tritt auch nur einmal auf und ist uns nicht zu erklären.