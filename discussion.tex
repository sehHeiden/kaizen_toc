% !TeX spellcheck = de_DE

\todo{Voruntersuchung: Recherche verwenden die Genanten Unternehmen wirklich Kaizen und TOC?}
Bei der Voruntersuchung hat nur ChatGPT 3.5 eine Anzahl von unternehmen gefunden, die eine oder beide Methoden verwenden. Die Verschlechterung bei ChatGPT 4 kann durch die Verwendung der \textit{BING}-Suche entstehen. Da die Suche nur auf Deutsch erfolgte, war die Quellenlage womöglich nicht ausreichend.
Dafür spricht, dass Toyota nur indirekt als Unternehmen gelistet wird, dass Kaizen verwendet. Indem erwähnt wird, dass Toyota personal Porsche in Kaizen schulte.


Im Zusammenhang mit \gls{llm} gibt es einige Studien \todo{Quelle?} die Produktivitätsgewinne untersuchen. Da nur ein einzelner Text und darüber hinaus keine Vergleichsbasis in Sinne eines menschlich geschrieben Textes erstellt wird. Darüber hinaus stellt sich die Frage, mit welchem Niveau an Vorwissen und Erfahrung beim Schreibenden des Vergleichstextes anwenden sollte. Im Idealfall werden Texte von verschiedenen Niveaus wie Schüler, Studenten, Berufstätige, Forschende gepromptet und geschrieben. Dabei dürfte der Promptende\todo{Wie schreiben} nicht die selbe Person sein, die einen Vergleichstext schreibt, um Leaks in der Form von Wissen (Zeitersparnis) und Erwartbarkeit des Ergebnisses auszuschließen.\todo{In welches Kapitel kommt dieser Absatz?}
