% !TeX spellcheck = de_DE

Ausgehend von den Voruntersuchungen mit ChatGPT 3.5 und ChatGPT 4, soll halb automatisiert ein Fachartikel erstellt werden,
der die Methoden Kaizen und \gls{toc} anhand eines Beispiels entlang der Prozesskette/Nutzung vergleicht~\cite{openai_chatgpt_nodate}. 
Der Text wird mit deutschen Prompts auf Deutsch erzeugt. Auf Basis von Voruntersuchungen mit ChatGPT 3.5 und 4, soll der Aufsatz dann mit ChatGPT entworfen werden. Inklusive der Forderung einer \LaTeX{}-Formatierung und Referenzen mit \BibTeX{} anzugeben.

Ziel ist es, dass die \gls{ki} möglichst viel an einem Stück entwirft, sodass Fragmente nicht zusammen getragen werden muss. Mikro-Management soll durch das Prompten von Texten für einzelne Absätzen vermieden werden. Da dies bei der Generierung aufwendiger ist auch mehr Fachwissen bedarf, die erst durch Vorarbeiten angeeignet werden sollen. Ggf. fügen wir dennoch Absätze des \gls{ki}-Entwurfs auf der Teilsatzebene zusammen. Wenn passend, werden auch die Teilsätze aus anderen Kapiteln übernommen. Bindewörter des Autoren sind durch eckige Klammern gekennzeichnet.

Danach wird dieser Text vollständig von einer Sprach in eine Zielsprache mit Deepl Pro übersetzt~\cite{deepl_deepl_nodate}.
Die einzelnen Übersetzungsschritte sind:

\begin{enumerate}
	\item Deutsch $\rightarrow$ Französisch
	\item Französisch $\rightarrow$ Russisch
	\item Russisch $\rightarrow$ Englisch
\end{enumerate}

Es handelt sich hierbei um eine qualitative Untersuchung. Es soll die Qualität des Entwurfs des deutschen Texts und die Fähigkeit der Übersetzungen ermittelt werden. Dies wurde gezielt so gewählt, um die Leistungsfähigkeit der Systeme zu fordern. Da weniger deutsche als englische Text im Internet verfügbar sind~\cite{noauthor_sprachen_nodate}, soll gezielt ein deutscher Text generiert werden unter der Prämisse, dass deutsche Text nach unserer Erfahrung inhaltlich schlechter ausfallen. Ein Textvergleich mit jeweils englischen und deutschen Prompts, wird aufgrund des Aufwands vermieden. Es sollen insgesamt gezielt Unzulänglichkeiten der \gls{llm} untersucht werden.

Es werden gezielt mehre Übersetzungsstufen verwendet, wie die Güte der Übersetzung je nach Sprachkombination variiert. Alle Prompts und Antworten, sowie alle Übersetzungen sind im Appendix angegeben. Werden Prompts korrigiert, sei es durch Fehler bei der Frage, oder weil die Fragestellung nicht zum wünschten Ergebnis führen, wird dieser Pfad nicht im Export des Chat und somit auch nicht hier im Anhang aufgeführt.
Der deutscher Text soll eine Ziellänge von vier Seiten bei gleicher Formatierung erreichen. Der englische Text kann dadurch etwas kürzer ausfallen. Der deutsche und der englische Text wird dieser Arbeit vorangestellt. 
Der englische Text wird auf formelle als auch inhaltliche Korrektheit auf dem Level eines wissenschaftlichen Fachaufsatzes überprüft. Der deutsche, der korrigierte französische und korrigierte russische Text wird dem Anhang beigefügt.
