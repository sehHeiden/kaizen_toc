% !TeX spellcheck = de_DE

Ausgehend von den Voruntersuchungen mit ChatGPT 3.5 und ChatGPT 4, soll halb automatisiert ein Fachartikel erstellt werden,
der die Methoden Kaizen und \gls{toc} anhand eines Beispiels entlang der Prozesskette/Nutzung vergleicht~\cite{openai_chatgpt_nodate}.
Der Artikel soll möglichst in einem Stück mit wenig wie nötig Prompts generiert werden. Das Mikro-Management durch das Prompten von Texten von einzelnen Absätzen soll vermieden werden. Da dies bei der Generierung aufwendiger ist auch mehr Fachwissen bedarf, die erst durch Vorarbeiten angeeignet werden sollen.
Der Text wird mit deutschen Prompts auf Deutsch mit Chatgpt 3.5/4 erzeugt. Danach wird dieser Text vollständig von einer Sprach in eine Zielsprache mit Deepl Pro übersetzt~\cite{deepl_deepl_nodate}.
Die einzelnen Übersetzungsschritte sind:

\begin{enumerate}
	\item Deutsch $\rightarrow$ Französisch
	\item Französisch $\rightarrow$ Russisch
	\item Russisch $\rightarrow$ Englisch
\end{enumerate}

Es handelt sich hierbei um eine qualitative Untersuchung. Es soll die Qualität des Entwurfs des deutschen Texts und die Fähigkeit der Übersetzungen ermittelt werden. Dies wurde gezielt so gewählt, um die Leistungsfähigkeit der Systeme zu fordern. Da weniger Deutsche als Englische Text im Internet verfügbar sind,\todo{Quelle?} soll gezielt ein deutscher Text geniert werden unter der Prämisse, dass ein deutsche Text nach unserer Erfahrung inhaltlich schlechter ausfallen. Es soll aber gezielt, nicht ein Deutscher und einer Englischer Text mit ähnlichen Deutschen und Englischen Prompts generiert werden, da dies dem Umfang der Arbeit überschreitet. Es sollen also gezielt Unzulänglichkeiten der \gls{llm} untersucht werden.
Es werden auch gezielt mehre Übersetzungsstufen verwendet, um zu erkennen, wo der deutsche Ursprungstext von der Englischen Zielübersetzung sich unterscheiden. Danach sollen untersucht werden in welchem Umsetzungsschritt die Abweichung entstand.
Alle Prompts und Antworten, sowie alle Übersetzungen sind im Appendix angegeben.
Der deutscher Text soll eine Ziellänge von vier Seiten bei gleicher Formatierung erreichen. Der englische Text kann dadurch etwas kürzer ausfallen und wird dieser Arbeit vorangestellt.
Der englische Text wird auf formelle als auch inhaltliche Korrektheit auf dem Level eines wissenschaftlichen Fachaufsatzes überprüft.
