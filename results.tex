% !TeX spellcheck = de_DE

\subsection{Beschreibung der Voruntersuchungen}

In den Voruntersuchungen wurden ChatGPT 3.5 (im Oktober 2023) und ChatGPT 4.0 (Im Januar 2024) angewiesen,
einen Aufsatz zum Vergleich von Kaizen und \gls{toc} zu entwerfen.

Bei der Untersuchung von ChatGPT 3.5~(siehe Anhang~\ref{sec:voruntersuch3}) wurde der Frageinhalt voll verstanden. ChatGPT weigerte sich, einen längeren Text zu entwerfen.
Bei der expliziten Fragen, welche Unternehmen beide Methoden anwenden, werden Toyota, General Electric, Procter \& Gamble aufgelistet.

Als ChatGPT 4 angewiesen wurde, einen Aufsatz zu entwerfen, wurde nach einen Grobentwurf nachgefragt, ob der Artikel in der Art geschrieben werden solle~(siehe Anhang~\ref{sec:voruntersuch4}). Bei der Nachfrage nach Firmen, die beide Techniken verwenden, wurde eine Netzsuche mit \textit{BING} gestartet. Dabei kam es zum Missverständnis, dass \gls{toc} als total organic carbon missverstanden wurde, was nur aus den Weblinks ersichtlich ist, die referenziert wurden. Daher konnten keine Firmen gefunden werden. Danach haben wir ChatGPT gebeten, Unternehmen in der Automobilbranche zu finden, die eine der beiden Methoden benutzten. Dabei wurde nur Porsche als Anwender von Kaizen gefunden. Danach wurde die \gls{ki} gebeten, a) mit einen hypothetischen Unternehmen in der Automobilbranche das beide Techniken verwende und  b) mit jeweils einem Unternehmen für Kaizen und \gls{toc}, einen Aufsatz zu entwerfen. In der Antwort gibt ChatGPT 4 an, dass es kein Unternehmen in der Automobilbranche findet, das \gls{toc} anwendet. Im Folgenden wird ein sehr kurzer Aufsatz zum hypothetischen Unternehmen angeboten und gefragt, ob damit der finale Text entworfen werden soll. Der Text ist nur eine der vier Seiten lang. Quellen werden nicht in einem Literaturverzeichnis, sondern als hochgestellte Weblinks angegeben.

\subsection{Beschreibung des KI-Textes: Erstellung und Inhalt}

Anhand der Voruntersuchungen soll nun mit ChatGPT 4 versucht werden, einen Aufsatz zum Vergleich von Kaizen und \gls{toc} zu erstellen. Die Unterhaltung mit ChatGPT 4 wurde vom 23.01.24 bis 25.01.24 in einem Chatverlauf auf drei Tage verteilt durchgeführt~(siehe Anhang \ref{sec:untersuch}).

Am ersten Tag wurde der allgemeine Aufbau erstellt. Am Tag darauf wurden weitere Antworten manuell in die am Vortag von der \gls{ki} entworfene \LaTeX{} ergänzt. Währenddessen wurden Referenzen in ein \BibTeX{}-Dokument zur Referenzierung gespeichert. Am dritten Tag wurde der Text durch Fallbeispiele ausgebaut. Das zusammengetragene Dokument wurde der \gls{ki} zur Kontrolle gegeben, welche keine Fehler in der Zitation und der manuellen Zusammenstellung der Textfragmente entdeckte. 

Das hieraus kompilierte PDF-Dokument wurde mittels DeepL übersetzt. 


\subsubsection{Die Erstellung des Originaltextes}

Beim ersten Prompt wurden die Firmen mitgegeben, die laut ChatGPT 3.5 sowohl Kaizen als auch \gls{toc} verwenden. Dabei sollte der Vergleich der beiden Methoden anhand eines Unternehmens entlang der Prozedur beschrieben werden. Der Aufsatz sollte eine Länge von vier Seiten haben. Die Quellen sollten explizit in allen möglichen Sprachen verfasst sein können. Der Text sollte als \LaTeX-Dokument mit Literaturverzeichnis geschrieben werden.

In der ersten Antwort legt sich ChatGPT 4 darauf fest, dass es keinen Zugriff auf Fachliteratur und eine Websuche hat. Daher legt die \gls{ki} Toyota als Firma fest, da hier der Einsatz beider Techniken allgemein bekannt sei.

In der selben Antwort wurde ein \LaTeX{}-Dokument von etwa einer Seitenlänge geschrieben.
Danach führt ChatGPT von selbst eine Suche durch und schreibt einen Text von einer Länge knapp unter zwei Seiten (ohne \LaTeX{}). Referenzen werden als Weblinks angegeben, aber ohne einen \BibTeX{}-String zu schreiben. Diese werden in folgenden Prompts immer gefordert.

Weiterhin bleibt die \gls{ki}, dabei keinen Zugriff auf Details und Statistiken zu haben. Bei der Nachfrage, den Text auf vier Seiten zu verlängern, wird ein neuer sehr ähnlicher Text gleicher Länge erstellt.

Im Allgemeinen werden Textfragmente, die sich hinreichend vom vorherigen Text unterscheiden, in die passenden Kapitel eingefügt. Wenn neue Textfragmente subjektiv besser erscheinen, ersetzen sie die vorherigen Fragmente, wie beispielsweise bei der nächsten Nachfrage, welche Techniken Toyota speziell einsetzt. Dies lieft einige neue Aspekte zu Kaizen im Allgemeinen, aber nicht zu den Methoden bei Toyota im Speziellen.
Danach suchen wir nach historischen Quellen zur Entstehung von Kaizen. Wir werden auf das Buch "`Kaizen: The Key to Japan's Competitive Sucess"' \cite{imai1986kaizen} verwiesen. Bei der Bitte dies im Text zu referenzieren, wechselt ChatGPT hier zum ersten Mal in einen direkten Zitierstil. Es seien aber keine Details zur Toyota enthalten. Hingegen sei der Text eher von allgemeiner Natur. Dabei weißt die \gls{ki} zum Ersten Mal hin, dass man sich auf Fallstudien beziehen könne. Bei zwei Fragen zum \BibTeX{}-String zum Buch werden zwei unterschiedliche Publisher verwendet. Wir verwenden nach einer Recherche (auch im Anhang) nur die zweite Variante. Bei der Frage nach Kaizen-Fallstudien bei Toyota werden zwei genannt~\cite{ibscmr2003toyotakaizen, casecentre2002toyotakaizen}. Fallstudien für \gls{toc} bei Toyota wurden nicht gefunden. Es wurde nur auf allgemeinere Quellen für \gls{toc} bei Toyota verwiesen, die wir noch eingefügt haben.
Zuletzt haben wir ChatGPT gebeten, 1) eine Einleitung zu Kaizen und \gls{toc} im Bezug auf Change Management zu erstellen 2) eine Zusammenfassung zu verfassen, 3) den manuell zusammengetragenen Text auf Richtigkeit, insbesondere beim Zitieren,  bewerten 4) zusätzliche Sätze in den Kapiteln "`Vergleich von Kaizen und \gls{toc}"' und Schlussfolgerung zu verfassen.

Davon getrennt wurden zwischendurch immer wieder Fragen zur Setzung mit \LaTeX{} gestellt, wie Randbreite, Autorenreferenzen, etc. ChatGPT legt Wert darauf, nicht als Autor genannt zu werden. Es nimmt dies jedoch nach Hinweisen auf wissenschaftliche Praktiken hin. Wir erwähnen die \gls{ki} daher im Autorenteil und der Zusammenfassung explizit, aber nicht als Autorïn.

Durch die Setzung mit \LaTeX{} kam es zu geringeren Setzungsfehlern, die sich nicht weiter auf die Übersetzungen auswirkten.

\subsubsection{Die Übersetzungen}

Die Übersetzungen wurden von DeepL ausgeführt. Die händisch redigierten Übersetzungen von Deutsch nach Französisch und Französisch nach Russisch sind im Anhang Kapitel "`Zwischenübersetzungen"' angefügt~\ref{sec:zwischenuebersetzung}. Die manuell redigierte, englische Übersetzung ist dieser Arbeit vorangestellt. 

Die Originaltexte wurde jeweils als PDF-Dokument hochgeladen. Das Ergebnis wurde als PDF-Dokument zurück gesendet. Das Dokument beginnt mit einem Wasserzeichen von DeepL, da wir die Kostenfreie Version nutzten.
Ansonsten ist auf dem ersten Blick nicht von einen menschlichen Text zu unterscheiden. Die Setzung des Textes, Schriftart und Größe, Absätze sind beinahe exakt gleich zum Original.
Einzig der Nachname des Autoren ist im Vergleich zum Vornamen immer leicht zu hoch gesetzt. Nachdem eine Seite im Originaltextdokument endet, wird in der neuen Übersetzung bis zum Ende der Seite übersetzt und dann, wie im Original eine neue Seite begonnen. Dies führt bei der Reihe von Übersetzungen, zu Seiten, die nur wenige Zeilen Text enthalten.

Im Folgenden betrachten wir nur Übersetzungsfehler, und weniger Grammatikfehler. Verschiebungen von Bedeutungen werden nicht betrachtet. 

Zum Vornamen des Autoren: dieser wurde in der französischen Version landestypisch angeglichen. Dadurch ist er in den folgenden Übersetzungen falsch. Der Nachnahme wurde im Russischen zu Hayden, was im Englischen beibehalten wurde.
Im Text selbst ist der größte Fehler der Wegfall eines ganzen Satzes ab der französischen Fassung. Kleinere Übersetzungfehler treten nur alle zwei Seiten auf. Weitere Auffälligkeiten sind gering. Wie zum Beispiel: Wenn ein Wort im deutschen Original am Zeilenende mit Bindestrich getrennt wurde, blieb der Bindestrich auch im Französischen erhalten auch wenn das Wort zusammen in eine Zeile passt.
Ab der russischen und insbesondere der englischen Übersetzung wird die Abkürzung \gls{toc} zum Teil falsch geschrieben. Der Name Baudin wird ab der russischen Version als Boden und das Initial M. als Herr in Russisch und Englisch übersetzt. Die größten Fehler sind aber in der englischen Übersetzung im Titel: "`...des Beispiels Toyota"' wurde "`...example of a company Toyota"' und aus dem Zusatz "`Unter Mitwirkung von ChatGPT"' im Autorenfeld wurde "`Courtesy of ChatGPT".