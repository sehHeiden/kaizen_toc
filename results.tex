% !TeX spellcheck = de_DE

\subsection{Beschreibung der Voruntersuchungen}

In den Voruntersuchungen wurden einmal ein ChatGPT 3.5 (im October 2023) und ChatGPT 4.0 (Im Januar 2024)
einen Aufsatz zum Vergleich von Kaizen und \gls{toc} zu entwerfen.

Bei der Untersuchung von ChatGPT 3.5~(siehe Anhang~\ref{sec:voruntersuch3}) wurde der Frageinhalt voll verstanden. ChatGPT weigerte sich einen längeren Text zu entwerfen.
Bei der Nachfrage welche Unternehmen Kaizen und \gls{toc} anwenden, wurden für Kaizen Toyota, General Electric, Ford, Nestlé  für \gls{toc}: Amazon, Intel, General Electric, Ford, Boing angegeben. Bei der expliziten Fragen, welche Unternhemen beides anwenden werden Toyota, General Electric, Procter \& Gamble aufgelistet.

Als ChatGPT 4, gefragt wurde einen Aufsatz zu entwerfen, wurde nach einen Grobentwurf nachgefragt, ob der Artikel in der Art geschrieben werden solle~(siehe Anhang~\ref{sec:voruntersuch4}). Bei der Nachfrage nach Firmen, die beide Techniken verwenden, wurde eine Netzsuche mit \textit{BING} gestartet. Dabei kam es zum Missverständnis, dass \gls{toc} als total organic carbon missverstanden wurde. Was nur aus dem Weblinks ersichtlich ist, die Referenziert wurden. Daher konnten, keine Firmen gefunden werden. Danach haben wir ChatGPT gebeten Unternehmen in der Automobilbranche zu finden, die eine der beiden Methoden benutzten. Dabei wurde nur Prosche gefunden. Danach wurde die \gls{ki} gebeten mit einen hypotetischen Unternehmen in der Automobilbranche das beide Techniken verwendet und jeweils einem Unternehmen, dass beide Verfahren verwendet, einen Aufsatz zu entwerfen. In der Antwort gibt ChatGPT 4, dass es kein Unternehmen in der Automobilbranche findet, dass \gls{toc} anwendet. Im folgenden wird ein sehr kurzer Ansatz mit nur noch dem Hypothetischen Fall angeboten und fragt, nach damit der Finale Text entworfen werden soll. Der Text ist nur eine der vier Seiten lang. Quellen werden nicht in einem Literaturverzeichnis, sondern als hochgestellten Weblink angegeben.

\subsection{Beschreibung des KI-Textes: Erstellung und Inhalt}

Mit diesem Voruntersuchungen, soll nun mit ChatGPT 4 versucht werden, einen Aufsatz zum Vergleich von Kaizen und \gls{toc} zu erstellen. Die Unterhaltung mit ChatGPT 4 wurde vom 23.01.24 bis 25.01.24 in einem Chatvervalauf auf drei Tage verteilt durchgeführt~\ref{sec:untersuch}. Werden Prompts korrigiert, sei es durch Fehler bei der Frage, oder weil die Fragestellung nicht nicht zum wünschten Ergebnis führen, wird dieser Pfad weder im Export des Chat und somit auch nicht hier im Anhang aufgeführt.

Am ersten Tag wurde der allgemeine Aufbau in der Diskussion durchgeführt. Am Tag darauf die Antworten manuell in ein von der \gls{ki} vorbereitetes \LaTeX{} ergänzt. Das Dokument wurde mit weiteren Fragen, das \LaTeX{}-Dokument in Stil und inhaltlich ausgebaut. Währenddessen Referenzen in ein \BibTeX{}-Dokument zur Referenzierung gespeichert. Am dritten Tag wurde das manuell ausgebaute Dokument der \gls{ki} zur Kontrolle gegeben. Es wurden keine Fehler in der Zitation und der manuellen Zusammenstellung der von \gls{ki} verfassten Textfragmente gefunden. Danach wurden von der letzte Abschnitte durch Fallbeispiele ausgebaut. 

Das hieraus kompilierte PDF-Dokument wurde an mittels DeepL übersetzt. 


\subsubsection{Die Erstellung des Originaltextes}

Bei ersten Prompt wurden die Firmen mitgegeben, die laut ChatGPT 3.5, sowohl Kaizen als auch \gls{toc} verwenden. Dabei solle der Vergleich der beiden Methoden anhand eines Unternehmens entlang der Prozedur beschrieben werden. Der Aufsatz solle eine Länge von vier Seiten haben. Die Quellen sollten explizit in allen möglichen Sprachen verfasst sein können. Der Text solle als \LaTeX-Dokument mit Literaturverzeichnis geschrieben worden sein.

In der ersten Antwort versteift sich die \gls{ki} darauf, dass sie keinen Zugriff auf Fachliteratur und eine Websuche zu haben. Daher versteifte sich die \gls{ki} auf Toyota als Firma, da hier der Einsatz beider Techniken allgemein bekannt sei.

In der selben Antwort wurde ein \LaTeX{}-Dokument von etwa einer Seite Länge geschrieben.
Danach führt ChatGPT von selbst eine Suche durch und schreibt eine Text von einer Länge knapp unter zwei Seiten (ohne \LaTeX{}). Referenzen werden als Weblinks angegeben, aber ohne einen \BibTeX{}-String zu schreiben.
\BibTeX
Weiterhin bleibt die \gls{ki} dabei keinen Zugriff auf Details und Statistiken zu haben.
Bei der Nachfrage, den Text auf vier Seiten zu verlängern, wird einen neuer sehr ähnlicher Text gleicher Länge erstellt.

Im allgemeinen werden die Textfragmente, die sich hinreichend vom vorherigen Text unterscheiden, in die passenden Kapitel hinzugefügt. Wenn neue Textfragmente subjektiv als besser erscheinen, ersezten sie die vorherigen Fragmente. So auch Beispielsweise bei der nächsten Nachfrage welche Techniken Toyota speziel einsetzt. Was einige neue Aspekte zu Kaizen im Allgemeinen aber nicht zu den Methoden bei Toyota im Speziellen liefert.
Danach suchen wir nach historischen Quellen zur Entstehen von Kaizen. Wir werden auf das Buch "Kaizen: The Key to Japan's Competitive Sucess"~\cite{imai1986kaizen} verwiesen. Bei der Bitte dies dies im Text zu referenzieren, wechselt ChatGPT hier zum ersten Mal in einen direkten Zitierstil. Es seien aber keine Details zur Toyota enthalten, sondern seien der Text eher allgemeiner Natur. Dabei weißt die \gls{ki} zum Ersten Mal hin, dass man sich auf Fallstudien beziehen möge. Bei zwei Fragen zum \BibTeX{}-String zum Buch, werden zwei Unterschiedliche Publisher verwendet. Wir verwenden nach einer Recherche (auch im Anhang) nur die zweite Variante. Bei der Frage zu Fallstudien nach Kaizen bei Toyota werden zwei genannt~\cite{ibscmr2003toyotakaizen, casecentre2002toyotakaizen}. Fallstudien für \gls{toc} bei Toyota wurden nicht gefunden.\todo{Kann ich welche finden?} Es wurde nur auf allgemeinere Quellen für \gls{toc} bei Toyota verwiesen, die wir noch eingefügt haben.
Zuletzt haben wir ChatGPT gebeten, 1) eine Einleitung zur Kaizen und \gls{toc} im Bezug auf Change Management 2) eine Zusammenfassung zu verfassen, 3) den manuell zusammengetragen Text auf Richtigkeit insbesondere beim Zitieren  bewerten 4) Zusätzliche Sätze in den Kapiteln "Vergleich von Kaizen und \gls{toc}"` und Schlussfolgerung zu verfassen.

Davon getrennt wurden zwischendurch immer wieder Fragen zur Setzung mit \LaTeX{} gestellt. Wie Randbreite, Autorenreferenzen, etc. ChatGPT sträubt sich gegen jede Autorenschaft. Nimmt sie jedoch mit hinweisen auf Wissenschaftliche Praktiken (dass man dies nicht tuen solle) an. Wir erwähnen daher im Autorenteil und der Zusammenfassung explizit, aber nicht als Autor.

Durch die Setzung mit \LaTeX{} kam es zu geringeren Setzungsfehlern. Die nicht weiter auf die Übersetzungen auswirkten.

\subsubsection{Die Übersetzungen}

Die Übersetzungen von wurden von DeepL ausgeführt. Die händisch redigierten Übersetzungen von Deutsch nach Französisch und Französisch nach Russisch sind im Anhang Kapitel Zwischenübersetzungen angefügt~\ref{sec:zwischenuebersetzung}. Die manuell redigierte, englische Übersetzung ist dieser Arbeit vorangestellt. 

Die Originaltext wurde jeweils als PDF-Dokument hochgeladen. Das Ergebnis wurde als PDF-Dokument zurück gesendet. Das Dokument beginnt mit einem Wasserzeichen von DeepL, da wir die Kostenfreie Version nutzten.
Ansonsten ist auf dem ersten Blick nicht von einen menschlichen Text zu unterscheiden. Die Setzung des Textes, Schriftart und Größe, Absätze sind beinahe exakt gleich zum Original.
Einzig der Nachname des Autoren im Vergleich zum Vornamen immer leicht zu hoch gesetzt. Wenn eine Seite im Originaltextdokument endet, wird in der neuen Übersetzung bis zum Ende der Seite übersetzt und dann, wie im Original eine neue Seite begonnen. Das führt bei der Reihe von Übersetzungen zu einer Erhöhung der Seitenzahl auf von fünf Seiten im Original auf acht in Englisch.

Im folgenden betrachten wir nur Übersetzungsfehler, und weniger Grammatikfehler. Verschiebungen von Bedeutungen werden nicht betrachtet. Kurz zum Namen des Autoren, dieser wurde in der französischen Version Landestypisch angeglichen. Dadurch ist er in den folgen Übersetzungen falsch. Der Nachnahme wurde im Russischen zu Hayden, was im Englischen beibehalten wurde.