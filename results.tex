% !TeX spellcheck = de_DE

\subsection{Beschreibung der Voruntersuchungen}

In den Voruntersuchungen wurden einmal ein ChatGPT 3.5 (im October 2023) und ChatGPT 4.0 (Im Januar 2024)
einen Aufsatz zum Vergleich von Kaizen und \gls{toc} zu entwerfen.

Bei der Untersuchung von ChatGPT 3.5(siehe Anhang~\ref{sec:voruntersuch3}) wurde der Frageinhalt voll verstanden. ChatGPT weigerte sich einen längeren Text zu entwerfen.
Bei der Nachfrage welche Unternehmen Kaizen und \gls{toc} anwenden, wurden für Kaizen Toyota, General Electric, Ford, Nestlé  für \gls{toc}: Amazon, Intel, General Electric, Ford, Boing angegeben. Bei der expliziten Fragen, welche Unternhemen beides anwenden werden Toyota, General Electric, Procter \& Gamble aufgelistet.

Als ChatGPT 4, gefragt wurde einen Aufsatz zu entwerfen, wurde nach einen Grobentwurf nachgefragt, ob der Artikel in der Art geschrieben werden solle (siehe Anhang~\ref{sec:voruntersuch4}). Bei der Nachfrage nach Firmen, die beide Techniken verwenden, wurde eine Netzsuche mit \textit{BING} gestartet. Dabei kam es zum Missverständnis, dass \gls{toc} als total organic carbon missverstanden wurde. Was nur aus dem Weblinks ersichtlich ist, die Referenziert wurden. Daher konnten, keine Firmen gefunden werden. Danach haben wir ChatGPT gebeten Unternehmen in der Automobilbranche zu finden, die eine der beiden Methoden benutzten. Dabei wurde nur Prosche gefunden. Danach wurde die \gls{ki} gebeten mit einen hypotetischen Unternehmen in der Automobilbranche das beide Techniken verwendet und jeweils einem Unternehmen, dass beide Verfahren verwendet, einen Aufsatz zu entwerfen. In der Antwort gibt ChatGPT 4, dass es kein Unternehmen in der Automobilbranche findet, dass \gls{toc} anwendet. Im folgenden wird ein sehr kurzer Ansatz mit nur noch dem Hypothetischen Fall angeboten und fragt, nach damit der Finale Text entworfen werden soll. Der Text ist nur eine der vier Seiten lang. Quellen werden nicht in einem Literaturverzeichnis, sondern als hochgestellten Weblink angegeben.

\subsection{Beschreibung des KI-Textes: Erstellung und Inhalt}

Mit diesem Voruntersuchungen, soll nun mit ChatGPT 4 versucht werden, einen Aufsatz zum Vergleich von Kaizen und \gls{toc} zu erstellen. 

