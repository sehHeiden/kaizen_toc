% !TeX spellcheck = de_DE

Insbesondere die Leistungsfähigkeit der Übersetzung hat uns überrascht. Die Dokumente sind rein äußerlich kaum vom Original zu unterscheiden. Die Fehlerfrequenz ist sehr gering, man kann sich dennoch nicht vollständig auf die Übersetzung verlassen. Es ist aber eine Erleichterung für Übersetzerïnnen und kann unter Vorbehalt, bei fehlender Sprachkenntnis verwendet werden.

Das Verhalten von ChatGPT 4 hat uns mehrfach überrascht, insbesondere wie sich deutsche Suchergebnisse auf das Ergebnis negativ auswirkten. Das Sprachmodel an sich funktioniert im Deutsch jedoch gut.

Dennoch eignet sich die \gls{ki} nicht zum schnellen Entwurf einer wissenschaftlichen Arbeit. Sie ist gut, um ein allgemeines Vorgehen zu planen, auch zum Schreiben von Einleitungen etc. kann man sie als Inspiration heranziehen. In späteren Kapiteln waren die Mängel zu groß, insbesondere, dass das System aktiv dagegen wirkt einen wissenschaftlichen Text selbst zu entwerfen, Quellen selbst zu suchen, oder zu verwenden. Ermöglicht wurde das zum Teil nur durch den Autoren und die Websuche der \gls{ki}.

Vielleicht könnte man eine wissenschaftliche Arbeit schreiben lassen, indem man das System zwingt in einen Kontrollzyklus zwingt, der das Ergebnis solange verändert, bis es die geforderten Qualitätskriterien und die Aufgabenstellung erfüllt. Dieses Verhalten kann man unter anderem bei der Programmierung erzwingen. Wenn ein Fehler im Programmcode der \glsadd{ki} die Ausführung verhindert, wird der Fehler analysiert und das Programm diesbezüglich verbessert.

Wann und wie ChatGPT 4 oder nachfolgende Modelle diese Vorgabe erfüllen, bleibt offen. Die eigentliche Zielstellung der Arbeit konnte durch die \gls{ki} nicht erfüllt werden. Der Ablauf beider Methoden wurde nicht akribisch verglichen, Fallstudien für \gls{toc} konnten nicht dargebracht werden.