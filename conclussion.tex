% !TeX spellcheck = de_DE

Insbesondere die Leistungsfähigkeit der Übersetzung hat uns überrascht. Die Dokumente sind rein äußerlich kaum vom Original zu unterscheiden. Die Fehlerfrequenz ist gering, aber dennoch sollte man die Sprachen beherrschen in die Übersetzt wird, da das Ergebnis bei weitem nicht fehlerfrei ist.

Das Verhalten von ChatGPT 4 hat uns auch Überrascht insbesondere wie sich Deutsche Suchergebnisse auf das Ergebnis negativ auswirkten. Das Sprachmodel als solches schon auch auf Deutsch von gut zu verwenden.

Dennoch eignet sich die \gls{ki} nicht zum schnellen Entwurf einer wissenschaftlichen Arbeit. Sie ist gut, um ein allgemeines vorgehen zu planen auch zum schreiben von Einleitungen etc. kann man sie als Inspiration heranziehen. Den späteren Kapiteln sind die Abweichungen vom soll groß. Insbesondere, dass sich das System aktiv dagegen wirkt einen wissenschaftlichen Text selbst zu entwerfen.

Vielleicht könnte man eine wissenschaftliche Arbeit schreiben lassen, in dem das System zwingt in ein Kontrolzyklus zu gelangen, indem solange jeweils die Arbeit verändert wird, bis es die gegeben Qualitätskriterien und Aufgabenstellung erfüllt. Wie auch bei insbesondere bei Schreiben von Programmen vorgeht, wenn es selber Kompilerfehler bei Ausführen von Pythoncode in der eigenen Laufzeitumgebung findet.

Wann und wie ChatGPT 4 oder nachfolgende Modelle, diese Vorgabe erfüllen bleibt offen.